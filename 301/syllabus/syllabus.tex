\documentclass[letterpaper,12pt]{article}
\nonstopmode

\usepackage{tgschola}
\usepackage[T1]{fontenc}
\usepackage[margin=.75in,letterpaper]{geometry}
\setlength{\parindent}{0in}
\usepackage{hyperref}
\usepackage{multicol}


\makeatletter
\renewcommand\section{%
  \@startsection {section}{1}{\z@}%
  {-2.5ex \@plus -1ex \@minus -.2ex}%
  {1.0ex \@plus.2ex\@minus-.2ex}%
  {\normalfont\Large\bfseries}}
\makeatother


\title{CSCI~301: Formal Languages and Functional~Programming\\
  \small ($5$ credits)}
\author{\small Western Washington University}
\date{\small Winter~2021}
\begin{document}
\maketitle

\begin{multicols}{2}
\begin{tabular}[t]{ll}
Instructor   & Aran Clauson \\
Office       & CF~411    \\
Office Hours & TBD       \\
Email        & Aran.Clauson@wwu.edu\\
Phone        & (360)~650-3819\\
\end{tabular}
\columnbreak
\begin{tabular}[t]{ll}
  TA    & Abby von Boeselager-Smith\\
  Email & vonboea@wwu.edu\\
  \\
  TA    & Quentin Jensen \\
  Email & jensenq@wwu.edu\\
\end{tabular}
\end{multicols}

\setlength{\parskip}{1ex}
\section*{Course Description}
 Introduction to discrete structures important to computer science, including
 sets, trees, functions, and relations. Proof techniques. Introduction to the
 formal language classes and their machines, including regular languages and
 finite automata, context free languages and pushdown automata. Turing machines
 and computability will be introduced. Programming using a functional language
 is required in the implementation of concepts. Includes lab.

{\small \textbf{Prerequisites:} CSCI~145}

\section*{Meeting Times}
\begin{center}
  \begin{tabular}{l|l|l|l}
    CRN   & Lecture  & Lab & Final \\ \hline
    13585 & MTWF 10:00--10:50am & Tue 8:00--9:50am   & Mon Mar~15 8:00--10:00am\\
    13586 & MTWF 10:00--10:50am & Fri 8:00--9:50am   & Mon Mar~15 8:00--10:00am\\
    14109 & MTWF 9:00--9:50am   & Tue 12:00--1:50pm  & Thr Mar~18 8:00--10:00am\\
    14110 & MTWF 9:00--9:50am   & Wed 10:00--11:50am & Thr Mar~18 8:00--10:00am\\
  \end{tabular}
\end{center}

Meetings will take place via Zoom.  A computer and Internet connection is
required.  Meeting URLs and passcodes will be posted to our Canvas
site:\url{http://canvas.wwu.edu/}.

\section*{Texts \& Resources}
This class is three parts: discrete math, computer science \emph{theory}, and
functional programming.  Discrete mathematics deals with sets, lists, and proof
strategies (e.g., proof by induction).  Computer science theory deals with
simplified, idealized machines and what they are capable of computing.  This is
almost always described as language acceptance problems.  We will learn the
\emph{Scheme} programming language.  It is a functional language and its
simplicity, power, and mathematical elegance will inform our study of computers
in the abstract, and it will also teach us new styles of programming.

\begin{itemize}\setlength{\parskip}{0ex}
\item \url{http://www.people.vcu.edu/~rhammack/BookOfProof/}
\item \url{http://cg.scs.carleton.ca/~michiel/TheoryOfComputation/}
\item \url{http://ds26gte.github.io/tyscheme/}
\item \url{http://racket-lang.org/}
\item \url{https://www.tug.org/texlive/}
\end{itemize}

Other resources will be made available throughout the term.  

\section*{Outcomes}
Upon completion of this course, students will demonstrate:
\begin{enumerate}\setlength{\parskip}{0ex}
\item
  Thorough understanding of the mathematical definitions of concepts important
  to computer science, including sets, tuples, lists, strings, languages,
  graphs, trees, functions and relations
\item
  The ability to prove basic theorems involving these mathematical concepts
\item
  The ability to employ effectively the functional programming style in a
  functional programming language
\item
  Solid understanding of fundamental classes of languages, including regular and
  context free, and their corresponding machines
\item
  Basic understanding of Turing machines and computability
\item
  Basic understanding of important algorithms, including conversion of finite
  automata to different forms, conversion of grammars to machines
\item
  Basic understanding of LL(k) and LR(K) grammars and the parsing techniques
  used for those grammars
\end{enumerate}

\section*{Grading Policies}
There will be 7 homework assignments, 8 lab assignments, one midterm, one final,
and a varying number of in-class exercises.  Assignments are weighted as
follows: 

\begin{center}
  \begin{tabular}[t]{ll}
    25\% & Homework \\
    25\% & Labs\\
    20\% & Midterm\\
    20\% & Final (required) \\
    10\% & In-Class participation\\
  \end{tabular}
\end{center}

Final grades are computed as follows:
\begin{verbatim}
(define grade 
  (lambda (pct)
    (cond ((> pct 0.90) #\A)
          ((> pct 0.80) #\B)
          ((> pct 0.70) #\C)
          ((> pct 0.60) #\D)
          (else         #\F)
    )
  )
)
\end{verbatim}
Pluses and minuses are assigned at my discretion.  The final exam is required! 


\section*{Tentative Schedule}

\begin{center}
  \begin{tabular}{l|ccc|l|l}
    & \multicolumn{3}{c|}{Reading} & & \\
    Week   & TYS   & BoP      & ITC & Labs  & Notes \\\hline
    Jan~4  & 1     & 1,2,3    &     &       &       \\
    Jan~11 & 2,3,4 & 4,5,6    &     & Lab 1 &       \\
    Jan~19 & 5,6,7 & 7,8,9,10 &  1  & Lab 2 &       \\
    Jan~25 &       & 11,12    &  2  & Lab 3 &       \\
    Feb~1  &       & 13,14    &  2  & Lab 4 &       \\
    Feb~8  &       &          &  3  & Lab 5 & Midterm Monday Feb~8\\
    Feb~15 &       &          &  3  & Lab 6 &       \\
    Feb~22 &       &          &  4  & Lab 7 &       \\
    Mar~1  &       &          &  5  & Lab 8 &       \\
    Mar~8  &       &          &     &       & Prep Week\\
  \end{tabular}
\end{center}
\textbf{TYS} is tyscheme, \textbf{BoP} is the Book of Proof, and \textbf{ITC} is
Introduction to Theory of Computation.  

\section*{Communication}
We will be using Canvas to post assignments and grades, but \textbf{do not} use
Canvas conversations to contact me.  I do not check Canvas for messages.  Please
use my email address, phone number, or stop by the office hours meeting.

\section*{Academic Honesty}
Academic dishonesty is not tolerated at Western Washington University. Someone
commits an act of academic dishonesty when he or she participates in
representing something as the work of a student that is not in fact the work of
that student. A Western student who is caught committing such an act at Western
typically fails the course in which it occurred, and repeated such acts can lead
to dismissal from the University. For a full description of the academic honesty
policy and procedures at Western, see Appendix D in the catalog.

\section*{Accommodations}
This course is intended for all WWU students, including those with visible or
invisible disabilities. Students with disabilities will be provided equitable
access to educational experiences and opportunities. If, at any point in the
quarter, you find yourself not able to fully access the space, content, and
experience of this course, please first contact the Disability Access Center
(DAC) to discuss potential accommodations. Faculty and staff partner with the
DAC in the implementation of accommodations.

If you already have accommodations set up through the DAC, please be sure to
send your Faculty Notification Letter the myDAC portal, and reach out to the DAC
so they can discuss how your approved accommodations apply to this course.

If you are unsure if accommodations are appropriate for you, contact the DAC for
more information, temporary assistance, or connections to other resources:
\texttt{https://disability.wwu.edu} or 360-650-3083.

\section*{Flexibility}
This syllabus is subject to change.  Changes, if any, will be announced in
class.  Students will be held responsible for all changes.

\section*{Religious Accommodation}

Western provides reasonable accommodation for students to take holidays for
reasons of faith or conscience or for organized activities conducted under the
auspices of a religious denomination, church, or religious
organization. Students seeking such accommodation must provide written notice to
their faculty within the first two weeks of the course, citing the specific
dates for which they will be absent.  ``Reasonable accommodation'' means that
faculty will coordinate with the student on scheduling examinations or other
activities necessary for completion of the course or program and includes
rescheduling examinations or activities or offering different times for
examinations or activities.  Additional information about this accommodation can
be found in SB 5166.

\section*{Student Rights \& Responsibilities Code}
In this course, students are held responsible for upholding all aspects of
Western's Academic Honesty Policy and Procedure, and the Student Rights and
Responsibilities Code.

\section*{Emergency Response}
Campus Emergencies: If during class an emergency arises in this classroom,
building or vicinity, your instructor may inform you of actions to follow to
enhance your safety. As a student in this class, you are responsible for knowing
emergency evacuation routes from this classroom. Know which of Western’s three
major disaster meeting locations is closest to this building (Old Main green,
grassy oval at Communications Facility or tennis courts).

Watch emergency videos and information at emergency.wwu.edu. If police or
university officials order the class to evacuate the classroom or building, do
so in a calm and orderly manner. Assist those who might need help in reaching a
barrier-free exit. You may receive Western Alert emergency information via the
building enunciation system or by text message, email, Facebook or Twitter.

\section*{Finals}

Finals Preparation Week (a.k.a. “Dead Week”): In preparation for Finals Week,
with some exceptions, exams are not to be administered and new graded
assignments shall not be introduced beyond the 5th week of the term that would
be due this week.

Finals Week: Final examinations, given in most courses at Western during the
last week of the quarter, are administered according to a Finals Schedule
(exceptions for lab courses). This generally differs from the usual class
meeting times. The scheduled days and hours for these examinations may not be
changed. For details about what happens when a student does not take a final,
see the Catalog. If students find they are scheduled to take three or more
examinations in one day, any of their instructors may arrange an examination
later during finals week.

\end{document}

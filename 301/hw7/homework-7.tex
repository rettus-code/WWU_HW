\documentclass{article}
\nonstopmode

\documentclass{article}
\nonstopmode

% Packages
% 
\usepackage{amsfonts,amssymb,amsmath}
\usepackage{tikz}
\usetikzlibrary{automata,positioning,arrows}
\usepackage[margin=1in]{geometry}

% Common Declarations
% 
% Make sure you change the author in common.tex
% 

\author{Michael Rettus}
\def\term{Winter 2021}
\def\course{CSCI~301}

% 
%
\setlength{\parskip}{1em}
\setlength{\parindent}{0in}

%
%
\newcounter{thequestion}
\newcommand{\question}{%
  \stepcounter{thequestion}%
  \noindent%
  \textbf{\arabic{thequestion}\quad}%
}
\newcommand{\answer}{%
  \par\noindent%
  \textbf{Answer\\}%
}



% Local Declarations
% 
\title{\course, \term\\Homework \#7}
\newcommand{\TO}{\ensuremath{\rightarrow}}
\newcommand{\OR}{\ensuremath{\quad|\:}}

% The document
% 
\begin{document}
\maketitle

\question%
Use the pumping lemma for context-free languages to prove that the following
language is not context-free: 
\[L = \left\{a^n:\text{$n$ is prime} \right\}\]
\answer
Assume for contradiction that L is a context-free language. We apply the
pumping lemma. Let m be the longest root to leaf length. Let p be a prime
that $p\ge m$. We pump the string $a^p \in L$. $a^p$= uvxyz, v = $a^j$ and y = $a^k$, with 
$j + k \ge 1$ since $|vy|\ge 1 $. From the pumping lemma we have that $uv^{p+1}xy^{p+1}z \in L$. \rightarrow $a^{p+jp+kp}$ \rightarrow $a^{p(1+j+k)} \in L$ which is a contradiction since $j + k \ge 1$, $p(1+j+k)$ is not a prime number\\\\
\question%
Use the pumping lemma for context-free languages to prove that the following
language is not context-free: 
\[L = \left\{w\in \left\{a,b\right\}^* : w=\left(ab^n\right)^n \: n\geq 0
    \right\}\]
\answer
Assume for contradiction that L is a context-free language. We apply the
pumping lemma. $\Rightarrow\exists$ a p for L, we choose w = $a^p,b^{p}^2$ which
= $(a,b^p)^p$ = $uv^ixy^iz$, $|vxy|\le p$ and $|vy|\ge 1$ therefore $p \ge 1$.\\\\
Case 1: $|vxy|$ is all a's and $i=1$ or $a^{p+1},b^{p}^2$ which is a contradiction
because unless p=0 $p^2$ cannot be the square of p+1.\\\\
Case 2: $|vxy|$ is all b's and $i=1$ or $a^p,b^{p^2+1}$ which is a contradiction
because $p^2+1$ cannot be the square of p for any number $\in \field{N}$.\\\\
Case 3: v is all a's, and y is all b's and $i=1$ or $a^{p+1},b^{p^2+1}$ which is a contradiction
because $p^2+1$ cannot be the square of p for any number $\in \field{N}$.\\\\
Case 4: v is some a's and some b's and y is all b's and $i=1$ the string of a's would stay p length followed by b's then a's $<$ p followed by $b^{p^s+1}$. Which is a contradiction and $\notin L$.\\\\
Case 5: v is all a's and y is some a's and some b's and $i=1$. The a's would increase to p+1 and the b's would stay $p^2$, but some b's followed by some a's both less than p would be between them and therefore $\notin L$ \\\\

\answer 2.0
after looking over my work I considered that since n is not an exponent in the normal sense that a different w was likely ie, if p=3 then $(ab^3)^3$ would look like abbbabbbabbb and different cases would be necessary. Plus I didn't want to delete all that work so there is that...\\
Case 1: p=1 therfore |vxy|=1 and |vy| = 1 in this case |vy| is a or b and if i=1 then by the pumping lemma then $w=(a^{p+1}b^p)^p$ or $(a^pb^{p+1})^p$ which is $\notin L$\\\\
Case 2: v is a and y is b and we let i=1 then $|vxy|=(a^2b^{p+1})$ which is $\notin w$\\\\
Case 3: v is b and y is a and we let i=1 then $|vxy|=(b^{p+1}a^2)$ which is $\notin w$\\\\
Case 4: $|vxy|$ is all b's and we let i=0 then $|vxy|=(a^2b^{p-|vy|})$ which is $\notin w$\\\\
Case 5: v contains a and some or t number of b's, where t $<$ p,  y is all b's and i=1. $|vxy|= (ab^t)^2b^{p-|x|-|t|+|y|}$ which is $\notin w$\\\\
Case 6: v contains all b's and y has an a plus some or t number of b's, where t $<$ p,  y is all b's and i=1. $|vxy|= b^{p-|x|-|t|+|v|}(ab^t)^2$ which is $\notin L$\\\\
\\
\question%
Consider the following grammar that generates the language of comma separated
lists of binary integers:
\begin{align*}
  G      &= \left(V, \Sigma, R, S \right) \\
  V      &= \left\{S, L, D, N \right\}\\
  \Sigma &= \left\{0,1,\textbf{,}\right\}\\
  R      &= \left\{
           \begin{array}{{r@{\:\rightarrow\:}l}}
             S & L | \epsilon                          \\
             L & N\text{\bf,}L | N                     \\
             N & ND | D                                \\
             D & 0 | 1  \\
           \end{array}
                 \right.\\
\end{align*}
This grammar is not LL(1).  look at the two rules for $L$ and $N$.

Change the grammar to make it LL(1).  You are welcome to add additional
variables, but you should only need two.  
\answer
% This is to make typesetting the grammar easier.  Modify the following LaTeX.  
\[
  \begin{array}{r@{\:\rightarrow\:}l}
    S & L        \\
    S & \epsilon \\
    N & DN'       \\
    N' & DN'|\epsilon    \\
    L & NL'      \\
    L'& ,L|\epsilon    \\
    L & N        \\
    D & 0        \\
    D & 1        \\  
  \end{array}
\]
\end{document}




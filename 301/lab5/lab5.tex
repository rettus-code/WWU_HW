\documentclass[letterpaper,12pt]{article}
\nonstopmode

% Packages
%
\usepackage[margin=1in]{geometry}
\usepackage{fancyvrb,amsmath}
\usepackage{multicol}

% Common Declarations
%
% Make sure you change the author in common.tex
%

\author{Michael Rettus}
\def\term{Winter 2021}
\def\course{CSCI~301}

% 
%
\setlength{\parskip}{1em}
\setlength{\parindent}{0in}

%
%
\newcounter{thequestion}
\newcommand{\question}{%
  \stepcounter{thequestion}%
  \noindent%
  \textbf{\arabic{thequestion}\quad}%
}
\newcommand{\answer}{%
  \par\noindent%
  \textbf{Answer\\}%
}


\fvset{frame=single,fontsize=\small}
\setlength{\columnsep}{2em}

% Local Declarations
% 
\title{CSCI 301, Lab \# 5}
\author{\term}

% The document
%
\begin{document}

\maketitle
\begin{multicols}{2}

  \paragraph{Goal:} This is the third in a series of labs that will build an
  interpreter for Scheme.  In this lab we will add the \verb|let| special form.

  \paragraph{Unit tests:}
  At a minimum, your program must pass the unit tests found in the
  file \verb|lab05-test.rkt|.  Place this file in the same folder
  as your program, and run it;  there should be no output.  Include
  your unit tests in your submission.

  \paragraph{Let:}
  
  The special form \verb|let| has the following syntax, with a typical example
  shown at right.  It consists of a list beginning with the symbol \verb|let|,
  then a list of symbol-value lists, and finally a single expression.  You will
  expand your \verb|evaluate-special-form| function to handle this case, and
  also add code to the \verb|special-form?| boolean to recognize a \verb|let|
  form.  No other changes need be made to your interpreter.
  \begin{multicols}{2}
\begin{Verbatim}
(let 
  ((sym1 exp1)
   (sym2 exp2)
   ...       )
  expr)
\end{Verbatim}
\begin{Verbatim}
(let 
  ((x (+ 2 2))
   (y x)
   (z (* 3 3)))
  (+ a x y z))
\end{Verbatim}
  \end{multicols}
  To evaluate this form, in an environment \verb|e1|, we evaluate all the forms,
  \verb|exp1|, \verb|exp2|, \ldots in the environment \verb|e1|.  Note that this
  means \verb|y| will get the value \verb|x| had in \verb|e1|, not 4.  Let's say
  that in \verb|e1| \verb|x| had the value 10 and \verb|a| had the value 20.
  
  We now have a list of variables, \verb|(x y z)|, and a list of values,
  \verb|(4 10 9)|.  We make a new environment by adding these variables and
  their values to the {\em front} of \verb|e1|.

  The single expression at the end of a \verb|let| form is now evaluated in this
  {\em new, extended} environment.  Thus, the final value will be
  \verb|(+ a x y z) => (+ 20 4 10 9) => 43|.  Make sure you understand this
  example before proceeding.

  It is also important to understand that this new, extended environment, is
  {\em only } used to evaluate the \verb|expr| embedded in the \verb|let| form.
  After the \verb|let| form is evaluated, you go back to using the old
  environment.  For example:
\begin{Verbatim}
(let ((x 10)) 
     (+ (let ((x 20)) (+ x x)) x))
=>  50
\end{Verbatim}
  In this example, we bind \verb|x| to 10, then create a local environment in
  which \verb|x| is bound to 20, in this new environment we evaluate
  \verb|(+ x x)| to get 40, and then, outside of the new environment we evaluate
  \verb|x| again, getting 10, which is added to 40 to get 50.

  A slightly trickier example is this:
\begin{Verbatim}[frame=single]
(let ((x 10)) 
   (+ (let ((x (+ x x))) (+ x x)) x))
\end{Verbatim}
  What do you think this will evaluate to?  Enter it into the Racket interpreter
  to see if you really understand.  Your interpreter should get the same result.

  In your implementation, all this tricky scoping will be handled by the fact
  that the extended environment in the \verb|let| form is {\em only} used to
  evaluate the included \verb|expr|.  It should not be visible outside of
  handling the \verb|let| form.

\end{multicols}
\end{document}

\documentclass{article}
\nonstopmode

% Packages
% 
\usepackage{amsfonts,amssymb,amsmath}
\usepackage[margin=1in]{geometry}

% Common Declarations
% 
% Make sure you change the author in common.tex
% 

\author{Michael Rettus}
\def\term{Winter 2021}
\def\course{CSCI~301}

% 
%
\setlength{\parskip}{1em}
\setlength{\parindent}{0in}

%
%
\newcounter{thequestion}
\newcommand{\question}{%
  \stepcounter{thequestion}%
  \noindent%
  \textbf{\arabic{thequestion}\quad}%
}
\newcommand{\answer}{%
  \par\noindent%
  \textbf{Answer\\}%
}



% Local Declarations
% 
\title{\course, \term\\Math Exercises \#3}
\newcommand{\nats}{\ensuremath{\mathbb{N}}}
\renewcommand{\theenumi}{\alph{enumi}}

% The document
% 
\begin{document}
\maketitle

\question%
%
Consider the relation $|$ (divides) on the set $\mathbb{Z}$.

\begin{enumerate}
\item Prove or disprove: $|$ is reflexive.
\answer \forall x $\in\mathbb{Z}$ $x=1x$, so $x|x$ and therefore reflexive.
\item Prove or disprove: $|$ is symmetric.
\answer Suppose x, y, z $\in\mathbb{Z}$ are such that $x|y=z$ so $y=zx$\\
$x=z|y$. Therefore $x|y\neq\ y|x$ $\forall\ z\in\mathbb{Z}$ where $z\neq\ |1|$\\
and is not symmetric.
\item Prove or disprove: $|$ is transitive.
\answer Suppose x, y, z $\in\mathbb{Z}$ are such that $x|y$ and $y|z$. Then a and b\\
are integers so that xa=y and yb=z. ab$\in\mathbb{Z}$ and x(ab)=z, so $x|z$.\\
The relationship is transitive.
\end{enumerate}

\question%
%
Assume $R$ and $S$ are two equivalence relations on a set $A$.

\begin{enumerate}
\item Prove or disprove: $R\cup S$ is reflexive.
\answer let $a\in\ A$ Since R and S are equivalent relations R and S are reflexive\\
So $(a,a)\in\ R$ and $(a,a)\in\ S$\\ $R\cup S$ contains all elements in either R or S.\\
$(a,a)\in\ R\cup S \forall a \in A$ and therefore $R\cup S$ is reflexive
\item Prove or disprove: $R\cup S$ is symmetric.

\begin{align*}
 Let (x,y) \in R\cup S\Rightarrow (x,y) \in R  &\bigvee (x,y) \in S \\
\Rightarrow (y,x) \in R  &\bigvee \Rightarrow (y,x) \in S \\
\Rightarrow (y,x) \in R\cup S  &\bigvee \Rightarrow (y,x) \in R\cup S\\   
\Rightarrow R\cup S symmetric &\bigvee \Rightarrow R\cup S symmetric
\end{align*} 
\item Prove or disprove: $R\cup S$ is transitive.
\answer A={a,b,c} and R and S are equivalent relations on A. Then let\\
\begin{align*}
    R &=\{(a,a),(b,b),(c,c),(a,b),(b,a)\}\\
    S &=\{(a,a),(b,b),(c,c),(b,c),(c,b)\}\\
\end{align*}
Then\\
\begin{align*}
    R \cup S &=\{(a,a),(b,b),(c,c),(a,b),(b,a),(b,c),(c,b)\}\\
\end{align*}
$a(R\cup S)b\wedge b(R\cup S)c\nRightarrow a(R\cup S)c$, thus $R\cup S$  is not transitive
\end{enumerate}

\question%
%
Consider the function $\theta : \{0,1\}\times\mathbb{N} \rightarrow \mathbb{Z}$
defined as $\theta(a,b) = a -2ab + b$

\begin{enumerate}
\item Prove or disprove: $\theta$ is injective.
\answer We show that $\theta$ is injective by contrapositive proof.\\ Suppose $\theta$(a,b) = $\theta$(c,d). then $a-2ab+b=c-2cd+d$\\
we know the first element of our ordered pair is either 0 or 1\\ Suppose a=c=1. Then\\
\begin{align*}
    a-2ab+b&=c-2cd+d\\ 1-2b+b&=1-2d+d\\ -2b+b&=-2d+d\\ -b&=-d\\ b&=d
\end{align*}
Suppose a=c=0. Then
\begin{align*}
    a-2ab+b&=c-2cd+d\\ 0-0+b&=0-0+d\\ b&=d
\end{align*}
Now suppose a=1 and c=0. Then
\begin{align*}
    a-2ab+b&=c-2cd+d\\ 1-2b+b&=0-0+d\\ 1-b&=d
\end{align*}
so $\theta(a,b)=\theta(c,d)\Rightarrow(a,b)=(b,d) $ therefore $\theta$ is injective.
\pagebreak
\item Prove or disprove: $\theta$ is surjective.
\answer Disprove that $\theta$ is surjective by direct proof.
\begin{align*}
 f(a) &= b\\ a - 2ab + b &= 0\\ -b(2a-1) &=-a\\ b&=a/(2a-1)
\end{align*}
b $\in Z$ where f(a)=b=a-2ab+b. Now suppose f(a)=2
\begin{align*}
    a-2ab+b&=0\\ a-4a+2&=0\\ -3a&=-2\\ a&=2/3
\end{align*}
$2/3 \notin$ N and therefore the function is not surjective.
\end{enumerate}

\end{document}




\nonstopmode
\documentclass[12pt]{article}

% Pacakges
%
\usepackage[margin=1in]{geometry}
\usepackage{amsmath,amssymb}

% Common Declarations
%
% Make sure you change the author in common.tex
%

\author{Michael Rettus}
\def\term{Winter 2021}
\def\course{CSCI~301}

% 
%
\setlength{\parskip}{1em}
\setlength{\parindent}{0in}

%
%
\newcounter{thequestion}
\newcommand{\question}{%
  \stepcounter{thequestion}%
  \noindent%
  \textbf{\arabic{thequestion}\quad}%
}
\newcommand{\answer}{%
  \par\noindent%
  \textbf{Answer\\}%
}



% Local Declarations
%
\title{\course, \term\\Math Exercises \#2}

\newcommand{\nats}{\ensuremath{\mathbb{N}}}

% The document
%
\begin{document}
\maketitle


\begin{itemize}
\item
Prove each of the following statements.  Explicitly state which method
of proof you are using: direct, contrapositive, contradiction,
induction, {\em etc.}
\item
Format all solutions in \LaTeX.
Please note any time you use {\bf Sympy} or
other software to simplify algebraic expressions, and include the
transcript of the interaction.
\end{itemize}


\question%
If two integers have the same parity, then their sum is even.
\answer by direct proof suppose x and y are odd and d and e are even \\
x=2a+1 for some $a\in\mathbb{Z}$ and y=2b+1 for some $a\in\mathbb{Z}$, by definition of an odd number and\\
d=2a for some $a\in\mathbb{Z}$ e=2b for some $a\in\mathbb{Z}$, by definition of an even number\\
by definition of same parity both numbers are odd or both numbers are even\\
x+y=2a+1+2b+1=2a+2b+2=2(a+b+1), a+b+1=c for some $c\in\mathbb{Z}$ x+y=2c\\ 
d+e=2a+2b=2(a+b), a+b=c for some $c\in\mathbb{Z}$ d+e=2c\\ 
so x+y and d+e by definition are even.\\
%


\question%
Suppose $a\in\mathbb{Z}$.  If $a^2$ is not divisible by 4, then $a$ is odd.
\answer contrapositive proof statement: if a is even then $a^2$ is divisible by 4.\\
Suppose n is even\\
So n=2a for some $a\in\mathbb{Z}$ by definition of an even number\\
therefore $n^2=(2a)^2=4a^2$\\
So $n^2$ is divisible by 4 making the contrapositive statement true\\
Based on that proof if $a^2$ is not divisible by 4 is true then a is odd is also true.
%
\newpage 
\question%
Suppose $a,b\in\mathbb{Z}$.  If $4\mid (a^2+b^2)$, then $a$ and $b$ are not both
odd.
\answer contrapositive proof: Suppose $4\mid (a^2+b^2)$, and both $a$ and $b$ are odd.\\
let a=2m+1 and b=2n+1 for some m and n $\in\mathbb{Z}$\\
$a^2+b^2=(2m+1)^2+(2n+1)^2=4m^2+4m+4n^2+4n+2=4(m^2+m+n^2+n)+2$\\
Then $4\nmid (a^2+b^2)$%

  
\question%
For all $n\in\nats$:
\begin{align*}
  \sum_{i=1}^n i(i+1) &= \frac{n(n+1)(n+2)}{3} \\
                     &= \frac{1}{3}\times
                       \left(n\left(n+1\right)\left(n+2\right)\right) \\
\end{align*} 
\answer%
Base case n=1   
\begin{align*}
                1(1+1)&= \frac{1}{3}\times
                       \left(1\left(1+1\right)\left(1+2\right)\right) \\
                  2  &= 2\\
\end{align*}
Induction hypothesis: Assume that for some $n\ge1$ we have
\begin{align*}
   \sum_{k=1}^n k(k+1) &= \frac{n(n+1)(n+2)}{3} \\
   \sum_{k=1}^{n+1} k(k+1)&= \left(n+1\right)\left(n+2\right)\times\frac{1}{3}
                       \left(n\left(n+1\right)\left(n+2\right)\right) \\
                       &=\frac{1}{3}\left(n+1\right)\left(n+2\right)\left(n+3\right)
\end{align*}
The formula holds for n, so it holds for n + 1. Therefore, the identity is true for all integers $n\ge1$.
% Notice how the equal signs are lined up in the generated PDF?  That is what
% the align* environment does; it aligns the symbol after the ampersand (&).  
%


\newpage
\question%
For all $n\in\nats$:
\[
  \sum_{i=1}^n (8i-5) = 4n^2 - n  
\]
\answer Let F(n) be: $3+11+19+...+(8n - 5)=4n^2-n$\\%
Base case n=1   
\begin{align*}
                (8(1)-5)&=  4(1)^2 - 1\\
                  3  &= 3\\
\end{align*}
  Induction hypothesis: Assume F(k) is true $[(8k-5)=4k^2-k]$ for F(k+1)
\begin{align*}
        (8k-5)+(8(k+1)-5)&=4(k+1)^2−(k+1)\\
         4k^2-k+8k+3 &= 4(k^2+2k+1)-k-1\\
         4k^2+7k+3 &= 4k^2+7k+3\\
\end{align*}  
F(k+1) is true when F(k) is true\\
\\
\question%
For all $n\in\nats$:
\[
  9\mid (4^{3n} + 8)
\]
\answer%
Base case n=1   %
\begin{align*}
                &9\mid (4^{3(1)} + 8)\\
                &9\mid 72 = 8\\
\end{align*}
Induction step assume $9\mid (4^{3n} + 8)$ is true for $n\ge1$ then $4^{3n} + 8=9c$ for some $c\in\mathbb{Z}$: \\
\\
$4^{3(n+1)} + 8 = 4^3\cdot4^{3n} + 8 = 64\cdot4^{3n} + 8 = 63\cdot4^{3n} + (4^{3n} + 8) = 9(7\cdot4^{3n} + p)$\\

$9\mid (4^{3(n+1)} + 8)$ by induction holds up for any $n\ge0$ where $n\in\nats$ 
\end{document}

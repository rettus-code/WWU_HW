\documentclass[letterpaper,12pt]{article}
\nonstopmode

% Packages
%
\usepackage[margin=1in]{geometry}
\usepackage{fancyvrb,amsmath}
\usepackage{multicol}
\usepackage{xcolor}

% Common Declarations
%
% Make sure you change the author in common.tex
%

\author{Michael Rettus}
\def\term{Winter 2021}
\def\course{CSCI~301}

% 
%
\setlength{\parskip}{1em}
\setlength{\parindent}{0in}

%
%
\newcounter{thequestion}
\newcommand{\question}{%
  \stepcounter{thequestion}%
  \noindent%
  \textbf{\arabic{thequestion}\quad}%
}
\newcommand{\answer}{%
  \par\noindent%
  \textbf{Answer\\}%
}


\fvset{frame=single}


% Local Declarations
% 
\title{CSCI 301, Lab \# 7}
\author{\term}
\date{}

% The document
%
\begin{document}
\maketitle

\paragraph{Goal:} This is the last in a series of labs that will
build an interpreter for Scheme.  In this lab we will build our own
parser for Scheme expressions using a simple LL(1) grammar to guide
us. 

\paragraph{Unit tests:}
At a minimum, your program must pass the unit tests found in the file
\texttt{lab8-test.rkt}.  Place this file in the same folder as
\texttt{lab8.rkt}, and also \texttt{lab7.rkt}, and run it; there should be no
output.  Include your unit tests and both labs in your submission.

\paragraph{From strings to expressions.}

Up until now we let {\bf Racket} handle the conversion from characters to lists,
that is, changing the characters you typed into a file, for example,
\verb|'(+ 1 2)|, into an actual Scheme list structure with \texttt{car}s and
\texttt{cdr}s.  \textbf{Racket} has a builtin expression reader which does this
automatically.

Now we are going to do that part ourselves.  Our unit test file, for example,
can look like this:
\begin{Verbatim}[frame=single]
(require rackunit "lab8.rkt" "lab7.rkt")
(define e2 (list (list '+ +)))
(check equal?
       (parse "(+ 1 2)")
       '(+ 1 2))
(check equal?
       (evaluate (parse "(+ 1 2)") e2)
       3)
\end{Verbatim}
Note that your interpreter will be the one defined in \texttt{lab7.rkt}, and
will provide \texttt{evaluate}, and the parser will be defined in
\texttt{lab8.rkt} and provide \texttt{parse}.  The parser and the evaluator
should be completely independent.


\paragraph{Input}
We will use the same strategy for input that we used in the RPN calculator.
Thus, a string will be converted to a list of characters.  That list will be
passed between functions representing the variables of the grammar.  The
functions will consume the input list and produce a list that can be passed to
\texttt{evaluate}.  

\paragraph{LL(1) parsing for Scheme}

The grammar for our Scheme is incredibly simple:
\begin{align*}
  L &\rightarrow \textvisiblespace L | EL | \epsilon  &\mbox{Expression List}\\
  E &\rightarrow DN \mid AS \mid (\ L\ ) &\mbox{Expression}\\
  S &\rightarrow AS \mid \epsilon &\mbox{Symbol}\\
  N &\rightarrow DN \mid \epsilon &\mbox{Number}\\
  D &\rightarrow 0 \mid 1 \mid 2 \mid 3 \mid ... &\mbox{Digit}\\
  A &\rightarrow a \mid b \mid c \mid d \mid ... &\mbox{Symbolic}\\
\end{align*}
%
There are a couple of things to note with the grammar shown above.  First, the
visible-whitespace character is ``\textvisiblespace''.  Second it is not,
strictly speaking LL(1); we will fix this in the first and follow sets.
Finally, as in our RPN calculator, we used some builtin Scheme procedures to
classify characters.  For example, \texttt{char-numeric?} will identify digits,
\texttt{char-whitespace?} will identify whitespace.

For identifying characters we can use in a symbol (called ``Symbolic'' in the
grammar above), we want to include just about everything except whitespace and
the two parentheses.  This is easy to define:
\begin{Verbatim}[frame=single]
(define char-symbolic?
  (lambda (char)
    (and (not (char-whitespace? char))
         (not (eq? char #\())
         (not (eq? char #\))))))  
\end{Verbatim}
We are interested in five kinds of characters: \verb|#\(|, \verb|#\)|,
\verb|char-numeric?|, \verb|char-symbolic?|, and~\verb|char-whitespace?|


\begin{center}
  \begin{tabular}[t]{l|l|l|l}
    &Null&First&Follow\\\hline
    L&yes&\{(,\textvisiblespace,0..9,a..\}&\{)\$\}\\
    E&no &\{(,0..9,a..\}&\{(,),\textvisiblespace,0..9,a..,\$\}\\
    S&yes&\{a..\}&\{(,),\textvisiblespace,0..9,\textbf{a..},\$\}\\
    N&yes&\{0..9\}&\{(,),\textvisiblespace,\textbf{0..9},a..,\$\}\\
    D&no &\{0..9\}&\{(,),\textvisiblespace,0..9,a..,\$\}\\
    A&no &\{a..\}&\{(,),\textvisiblespace,0..9,a..,\$\}\\
  \end{tabular}
\end{center}

Notice how S and N are nullable and that the intersection of their First and
Follow sets are not empty?  This means that the grammar is not LL(1).

We are faced with two options: make the grammar more complicated or directly
modify the First and Follow sets.  Let's modify the sets.

By removing the common symbols from the follow set, the resulting parser will
generate the longest symbol or number.  That is, a digit cannot directly follow
an N.  Similarly, a symbol character cannot follow an S.  The modified sets look
like this:

\begin{center}
  \begin{tabular}[t]{l|l|l|l}
    &Null&First&Follow\\\hline
    L&yes&\{(,\textvisiblespace,0..9,a..\}&\{)\$\}\\
    E&no &\{(,0..9,a..\}&\{(,),\textvisiblespace,0..9,a..,\$\}\\
    S&yes&\{a..\}&\{(,),\textvisiblespace,0..9,\$\}\\
    N&yes&\{0..9\}&\{(,),\textvisiblespace,a..,\$\}\\
    D&no &\{0..9\}&\{(,),\textvisiblespace,0..9,a..,\$\}\\
    A&no &\{a..\}&\{(,),\textvisiblespace,0..9,a..,\$\}\\
  \end{tabular}
\end{center}


Write a parser following the same pattern as we did for the RPN calculator.
There are only a couple of significant differences between the calculator and
this parser.

The $E\rightarrow (L)$ rule creates a nested list.  This is similar to the way
that \verb|rpn-parser.rkt| handled numbers.  This was wrong for numbers, but is
right for lists.  

The symbols in the RPN parser were a handful of single characters.  Symbols in
scheme must be at least a single character, but can be longer.  They cannot have
embedded numbers nor can they have embedded parentheses.  You might have to do
something similar to how we handled integers in the calculator parser.  


\paragraph{Read-Eval-Print Loop: optional}
The read-eval-print loop is the main body of an interactive interpreter.  
\begin{Verbatim}
(define repl (lambda ()
    (display (evaluate (parse (read-line))))
    (newline)
    (repl)
    )
  )
\end{Verbatim}

\paragraph{Program files: optional}

Note that our parser only handles strings typed in, but that we
usually want an interpreter to interpret program files, not strings.
The solution is pretty trivial:
check out the {\bf Racket} procedure {\tt file->string}.

With a little work you could get your interpreter to interpret itself...


\end{document}


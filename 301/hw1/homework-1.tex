\nonstopmode
\documentclass[12pt]{article}

% Pacakges
%
\usepackage[margin=1in]{geometry}
\usepackage{amsmath,amssymb}
\usepackage{fouriernc}
\usepackage[T1]{fontenc}

% Common Declarations
%
% Make sure you change the author in common.tex
%

\author{Michael Rettus}
\def\term{Winter 2021}
\def\course{CSCI~301}

% 
%
\setlength{\parskip}{1em}
\setlength{\parindent}{0in}

%
%
\newcounter{thequestion}
\newcommand{\question}{%
  \stepcounter{thequestion}%
  \noindent%
  \textbf{\arabic{thequestion}\quad}%
}
\newcommand{\answer}{%
  \par\noindent%
  \textbf{Answer\\}%
}



% Local Declarations
%
\title{\course, \term\\Math Exercises \#1}

% The document
%
\begin{document}
\maketitle

% 
%
\begin{itemize}
\item Answer all of the questions.
\item Leave the questions in place; it makes grading easier. 
\item Add and commit your \texttt{homework-1.tex} to your git repository.
  \textbf{Do not} add and commit generated files like \texttt{homework-1.pdf}
  or \texttt{homework-1.log}.  Consider a \texttt{make clean} before using git.  
\item Show your work and explain your answer.  If the answer is an integer, I
  need to know how you got it.
\item Remove this itemized list.
\end{itemize}

\question Explicitly write out the contents of the following set:
\[ \left\{X\in \mathcal{P}\left(\left\{1,2,3\right\}\right) : 2\in X\right\}\]

\answer%
\emph{X is an element of the power set of set \{1,2,3\} where 2 is an element of X}

\question
Negate the following statement:\\
If $x$ is a rational number and $x\neq 0$, then $\tan(x)$
is not a rational number.

\answer%
\emph{$\sim(x\in Q)\cup(x=0) \Rightarrow tan(x)\in Q$}
  

\question Compute how many 7-digit numbers can be made from the digits
$1,2,3,4,5,6,7$ if there is no repetition and the odd digits must appear in an
unbroken sequence.  Examples: $3571264$ or $2415376$ or $2467315$, but not
$7234615$.

\answer%
\emph{\{1,3,5,7\}is 4!,\{\{1,3,5,7\},2,4,6\} is also 4! so $4!^2 = 576$}
  
\question This problem concerns 4-card hands dealt off a standard 52-card deck.
How many 4-card hands are there for which all 4 cards are of different suits or
all 4 cards are red?

\answer%
\emph{let S be the cards of different suits and R be the red cards. \\ $S\cap R = \emptyset$}
\emph{\\So \\ $|S + R|=|S|+|R|$\\and \\$|S| = 13^4$, while $|R| = {26 \choose 4}$ \\$13^4 + {26 \choose 4}=43511$}  
\end{document}

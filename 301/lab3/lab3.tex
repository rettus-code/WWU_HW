\documentclass[letterpaper,12pt]{article}
\nonstopmode

% Packages
%
\usepackage[margin=1in]{geometry}
\usepackage{fancyvrb,amsmath}
\usepackage{multicol}

\newcommand{\set}[1]{\ensuremath{\{#1\}}}
\newcommand{\power}[1]{\ensuremath{\mathcal{P}(#1)}}

% Common Declarations
%
% Make sure you change the author in common.tex
%

\author{Michael Rettus}
\def\term{Winter 2021}
\def\course{CSCI~301}

% 
%
\setlength{\parskip}{1em}
\setlength{\parindent}{0in}

%
%
\newcounter{thequestion}
\newcommand{\question}{%
  \stepcounter{thequestion}%
  \noindent%
  \textbf{\arabic{thequestion}\quad}%
}
\newcommand{\answer}{%
  \par\noindent%
  \textbf{Answer\\}%
}



% Local Declarations
% 
\title{\course, Lab \# 3}
\author{\term}
\date{}

% The document
%
\begin{document}

\maketitle
\setlength{\columnsep}{2em}

\begin{multicols}{2}

\paragraph{Goal:} This lab begins a series of labs that will
build an interpreter for Scheme.  In this lab we get used to
recursion as an evaluation strategy, and programming it.
Our programs will be represented by lists, to avoid parsing problems
until later.

\paragraph{Unit tests:}
At a minimum, your program must pass the unit tests found in the file {\tt
  lab3-test.rkt}.  There should be no output.  Include your unit tests in your
submission.

\paragraph{Evaluation:}  
The process of evaluating a Scheme expression that consists only of
function calls (procedure applications), for example
{\tt (+  3 x (+ 2 2) (* 2 (+ 4 1)))}, is really very simple.
The process follows three simple rules:
\begin{enumerate}\setlength{\itemsep}{0pt}

\item Numbers evaluate to themselves.

  To check whether something
  is a number in scheme, use the {\tt number?} predicate.

\item Symbols, such as {\tt cons}, {\tt +}, and {\tt x},
  are looked up in a table
  called the {\em environment}.  We will use a list for our
  environment.

  To check whether something is a symbol in scheme, use the
  {\tt symbol?} predicate.  Do you see a pattern here?

\item Lists are evaluated recursively.  First, each element of the list is
  evaluated, and then the first argument ,which should be a procedure at this
  point, is applied to the evaluated arguments:
  
  {\hspace{-\leftmargin}
    \footnotesize \tt 
    \begin{tabular}{c@{}c@{\;}c@{\;}c@{\;}c@{\;}c@{}c}
      ( & + & 3 & x & (+ 2 2) & (* 2 (+ 4 1)) & )\\
        & $\Downarrow$& $\Downarrow$& $\Downarrow$& $\Downarrow$& $\Downarrow$& \\
      ( & \#<procedure:+> & 3 & 5 & 4 & 10 & )\\
        &$\Downarrow$ \\
        &22\\
    \end{tabular}
  }
  
\end{enumerate}

To implement this, you will write at least two procedures: {\tt lookup} and
{\tt evaluate}.

\paragraph{Lookup:}
The procedure {\tt lookup} is simple.  We represent an environment as a list of
lists.  Each sublist holds a variable, and the value of that variable. For
example:
\begin{Verbatim}[frame=single]
(define env (list (list 'x 5)
                  (list '+ +)
                  (list '* *)))
\end{Verbatim}
This would be enough of an environment to evaluate the above expression.

The {\tt lookup} procedure takes two arguments: a symbol and an environment, and
\begin{itemize}\setlength{\itemsep}{0pt}
\item If the first argument is not a symbol, returns an error.
\item If the symbol is not in the environment, returns an error.
\item Otherwise, returns the value bound to the variable.
\end{itemize}
{\tt lookup} should be written as a simple recursion through the
environment, comparing the provided symbol with the symbol in the
{\tt car} or {\tt first}
of each variable-value sublist.  If it finds a matching
symbol, the {\tt cadr} or {\tt second} thing is returned.
  
\paragraph{Evaluate:}
The {\tt evaluate} procedure takes two arguments, an {\tt expression} to
evaluate, and an {\tt environment}.  It follows the above rules:
\begin{itemize}\setlength{\itemsep}{0pt}

\item A number is returned unchanged.

\item For a symbol the return value is looked up in the environment.

\item For a list, each element in the list is evaluated recursively by
  the {\tt evaluate} procedure.  You may
  want to use {\tt map}, but it is not required.
  \begin{itemize}\setlength{\itemsep}{0pt}

  \item If the first thing in the evaluated list is not a procedure, an error is
    returned.  How do you think you check for a procedure in Scheme?  Don't
    overthink it.

  \item Otherwise, the procedure is applied to the evaluated arguments.

  \end{itemize}

\item If the expression is anything else, an error is returned.
\end{itemize}
\end{multicols}

\end{document}


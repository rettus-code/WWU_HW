\documentclass[letterpaper,12pt]{article}
\nonstopmode

% Packages
%
\usepackage[margin=1in]{geometry}
\usepackage{fancyvrb,amsmath}

% Common Declarations
%
% Make sure you change the author in common.tex
%

\author{Michael Rettus}
\def\term{Winter 2021}
\def\course{CSCI~301}

% 
%
\setlength{\parskip}{1em}
\setlength{\parindent}{0in}

%
%
\newcounter{thequestion}
\newcommand{\question}{%
  \stepcounter{thequestion}%
  \noindent%
  \textbf{\arabic{thequestion}\quad}%
}
\newcommand{\answer}{%
  \par\noindent%
  \textbf{Answer\\}%
}


\fvset{frame=single}

% Local Declarations
% 
\title{CSCI 301, Lab \# 7}
\author{\term}

% The document
%
\begin{document}
\maketitle

\paragraph{Goal:} This is the fifth in a series of labs that will build an
interpreter for Scheme.  In this lab we will add the {\tt letrec} special form.

\paragraph{Unit tests:} At a minimum, your program must pass the unit tests
found in the file \texttt{lab7-test.rkt}.  Place this file in the same folder as
your program, and run it; there should be no output.  Include your unit tests in
your submission.

\paragraph{Letrec creates closures that include their own definitions:}
Consider a typical application of \texttt{letrec}:
\begin{Verbatim}
(letrec ((plus (lambda (a b) (if (= a 0) b (+ 1 (plus (- a 1) b)))))
         (even? (lambda (n) (if (= n 0) true (odd? (- n 1)))))
         (odd? (lambda (n) (if (= n 0) false (even? (- n 1))))))
  (even? (plus 4 5)))
\end{Verbatim}
\texttt{plus} is a straightforward recursive function.  \texttt{even?} and
\texttt{odd?} are mutually recursive functions, each one requires the other.

If we use \texttt{let} instead of \texttt{letrec}, we will evaluate the
\texttt{lambda} forms in the current environment, and none of the three
functions will be defined in that environment.  Each of the closures will
contain a pointer to an environment in which the recursive functions are
\emph{not} defined.  Thus, we cannot simply use \texttt{let}.

We want the closures to close over an environment in which \texttt{plus},
\texttt{even?} and \texttt{odd?} \emph{are} defined.  To do this, we will follow
this strategy.
\begin{enumerate}
\item To evaluate a \texttt{letrec} special form, we first run through the
  variable-value pairs in the \texttt{letrec} expresion as if it was a simple
  \texttt{let}.  In other words, we go ahead and create the closures with the
  {\em wrong} environment.  We will fix this later.
\item Anything else in a \texttt{letrec} is also handled in the \texttt{let}
  fashion, for example
\begin{Verbatim}
(let ((a 2))
   (letrec ((x (+ a a)))
      (+ x x)))
\end{Verbatim}
  will just return 8.  You can reuse your old \texttt{let} code from previous
  labs for this part.
\item In the course of evaluating a \texttt{let} expression, you created a
  mini-environment, and appended that to the current environment, to get a new
  environment.  We will need pointers to all three of these in what follows.
  For this writeup, I'm going to call them the \texttt{OldEnvironment}, the
  \texttt{MiniEnvironment}, and the \texttt{NewEnvironment}.  They are
  illustrated as follows:
\begin{Verbatim}
((x 5) (y 10) ...)                                     <= OldEnvironment
((plus ...) (even? ...) (odd? ...))                    <= MiniEnvironment
((plus ...) (even? ...) (odd? ...) (x 5) (y 10) ...)   <= NewEnvironment
\end{Verbatim}
\item At this point, the closures in \texttt{MiniEnvironment} contain pointers
  to \texttt{OldEnvironment}.  We need to change these to point to
  \texttt{NewEnvironment}.
\item If there are any closures in \texttt{OldEnvironment}, however, they are
  already correct, so we don't want to change them!

\item So, we need to loop through just the variable-value pairs in \texttt{
    MiniEnvironment}.  If any variables are bound to \emph{closures}, we change
  the {\em saved} environment pointer inside the closure to point to
  \texttt{NewEnvironment}.

  Make sure you loop through only the variable-value pairs in
  \texttt{MiniEnviroment}. Note that \texttt{NewEnvironment} includes both
  \texttt{MiniEnvironment} and \texttt{OldEnvironment}. So we \emph{don't} want
  to loop through all the closures in \texttt{NewEnvironment}.

\item Since we need to change a data structure that already exists, this is
  definitely {\em not} functional style programming.  In fact, lists in
  \textbf{Racket} are immutable!  So we \emph{cannot} use lists any more to
  represent closures.

  Look up the documentation in \textbf{Racket} on mutable lists.  You'll find
  procedures such as \texttt{mcar}, \texttt{mcdr}, and \texttt{ mcons}, which
  handle mutable lists just the way \texttt{car}, \texttt{cdr} and \texttt{cons}
  handle immutable lists.

  But you will also find procedures such as \texttt{set-mcar!} and
  \texttt{set-mcdr!} for changing existing lists into new ones.

  Using mutable lists allows us to change the implementation of closures so that
  we can change the environment inside:

\begin{Verbatim}[label=Mutable Closures]
(define closure
  (lambda (vars body env)
    (mcons 'closure (mcons env (mcons vars body)))))
(define closure?
  (lambda (clos) (and (mpair? clos) (eq? (mcar clos) 'closure))))
(define closure-env
  (lambda (clos) (mcar (mcdr clos))))
(define closure-vars
  (lambda (clos) (mcar (mcdr (mcdr clos)))))
(define closure-body
  (lambda (clos) (mcdr (mcdr (mcdr clos)))))
(define set-closure-env!
  (lambda (clos new-env) (set-mcar! (mcdr clos) new-env)))
\end{Verbatim}

  If you take the previous lab and replace the old closure implementation with
  this one, everything should work as before.  Now aren't you glad you respected
  the interface in the last lab?

  If your previous lab doesn't work with this new implementation of closures,
  fix it!

\item After we replace all the closures in \texttt{MiniEnvironment} with
  pointers to \texttt{NewEnvironment}, we can now evaluate the body of the
  \texttt{letrec} form, using \texttt{NewEnvironment}.  Yay!  Recursive
  functions!

\item Check the unit test file for some tricky examples!

  Can you believe you've written an interpreter for such a complex language?
\end{enumerate}

\end{document}

